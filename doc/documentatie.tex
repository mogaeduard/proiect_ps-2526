% ==============================================================================
% DOCUMENTAȚIE PROIECT PROBABILITĂȚI ȘI STATISTICĂ
% Grupele 242 și 244
% ==============================================================================

\documentclass[12pt,a4paper]{article}

% Pachete necesare
\usepackage[utf8]{inputenc}
\usepackage[romanian]{babel}
\usepackage{amsmath}
\usepackage{amssymb}
\usepackage{amsthm}
\usepackage{graphicx}
\usepackage{hyperref}
\usepackage{listings}
\usepackage{xcolor}
\usepackage{geometry}
\usepackage{fancyhdr}
\usepackage{tocloft}
\usepackage{float}

% Configurare pagină
\geometry{margin=2.5cm}
\setlength{\parindent}{0pt}
\setlength{\parskip}{6pt}

% Configurare header/footer
\pagestyle{fancy}
\fancyhf{}
\fancyhead[L]{Proiect PS - Grupa 242}
\fancyhead[R]{\thepage}
\renewcommand{\headrulewidth}{0.4pt}

% Configurare cod R
\lstset{
    language=R,
    basicstyle=\ttfamily\small,
    keywordstyle=\color{blue},
    commentstyle=\color{green!60!black},
    stringstyle=\color{red},
    numbers=left,
    numberstyle=\tiny\color{gray},
    stepnumber=1,
    numbersep=8pt,
    backgroundcolor=\color{gray!10},
    frame=single,
    breaklines=true,
    breakatwhitespace=true,
    showstringspaces=false
}

% Comenzi personalizate
\newcommand{\sectiune}[1]{\newpage\section{#1}}

% ==============================================================================
% DOCUMENT
% ==============================================================================

\begin{document}

% ------------------------------------------------------------------------------
% COPERTĂ
% ------------------------------------------------------------------------------

\begin{titlepage}
    \centering
    \vspace*{2cm}
    
    {\Huge\bfseries Proiect de Laborator\par}
    \vspace{0.5cm}
    {\LARGE Probabilități și Statistică\par}
    
    \vspace{2cm}
    
    {\Large Grupele 242 și 244\par}
    
    \vspace{3cm}
    
    {\large
    \textbf{Echipa:}\\[0.3cm]
    \begin{tabular}{ll}
        \textbf{Lider:} & [Ceausescu Ana-Carina] \\[0.2cm]
        \textbf{Membru:} & [Moga Eduard-Andrei] \\[0.2cm]
        \textbf{Membru:} & [Tirdea Dominic-Alexandru] \\
    \end{tabular}
    }
    
    \vfill
    
    {\large 
    Facultatea de Matematică și Informatică\\
    Universitatea din București\\[0.3cm]
    An universitar 2025-2026\\
    \today
    }
    
\end{titlepage}

% ------------------------------------------------------------------------------
% CUPRINS
% ------------------------------------------------------------------------------

\tableofcontents
\newpage

% ------------------------------------------------------------------------------
% DESCRIERE GENERALĂ
% ------------------------------------------------------------------------------

\section{Descriere Generală}
\label{sec:descriere}

\textit{Această secțiune este completată de lider (Persoana A).}

\vspace{1cm}

[Aici liderul descrie în 2-3 paragrafe obiectivul general al proiectului, metodele utilizate și structura de colaborare.]

Proiectul constă în rezolvarea a trei exerciții distincte care acoperă diferite aspecte ale teoriei probabilităților și statisticii:

\begin{itemize}
    \item \textbf{Exercițiul 1:} Simularea unui vector aleator uniform pe discul unitate
    \item \textbf{Exercițiul 2:} Aplicație Shiny pentru vizualizarea funcțiilor de repartiție
    \item \textbf{Exercițiul 3:} Problema acului lui Buffon și variante ale acesteia
\end{itemize}

Fiecare membru al echipei este responsabil pentru implementarea și documentarea completă a exercițiului alocat.

% ------------------------------------------------------------------------------
% EXERCIȚIUL 1
% ------------------------------------------------------------------------------

\sectiune{Exercițiul 1}
\label{sec:ex1}

\textit{Această secțiune este scrisă doar de Persoana A.}

\vspace{1cm}

\subsection{Obiectiv}
Obiectivul acestui exercițiu este simularea unui vector aleator $(X,Y)$ uniform distribuit pe discul unitate
$D(1)=\{(x,y)\in\mathbb{R}^2 \mid x^2+y^2 \le 1\}$, folosind două metode:
\begin{itemize}
    \item metoda acceptării și respingerii (accept-reject) din pătratul $[-1,1]^2$;
    \item metoda transformării în coordonate polare.
\end{itemize}
Pentru ambele metode generăm $N=1000$ puncte și comparăm media empirică a distanței față de origine cu valoarea teoretică.

\subsection{Aspecte Teoretice}

\subsubsection{Uniformitate pe pătrat}
Dacă $U_1, U_2 \sim \text{Unif}([-1,1])$ sunt independente, atunci densitatea comună este
\[
f_{U_1,U_2}(u_1,u_2)=f_{U_1}(u_1)\,f_{U_2}(u_2)=\frac{1}{2}\cdot\frac{1}{2}=\frac{1}{4},
\quad (u_1,u_2)\in[-1,1]^2,
\]
și $0$ în rest. Deci $(U_1,U_2)$ este uniform distribuit pe pătratul $[-1,1]^2$.

\subsubsection{Acceptare-respingere pentru discul unitate}
Generăm $(X,Y)$ uniform pe $[-1,1]^2$ și acceptăm punctul dacă $X^2+Y^2\le 1$.
Punctele acceptate sunt uniforme pe $D(1)$ deoarece condiționarea unei distribuții uniforme pe o submulțime produce distribuție uniformă pe acea submulțime.

\subsubsection{Distribuția lui $R$ și $\theta$ (metoda polară)}
Pentru $(X,Y)$ uniform pe discul unitate, în coordonate polare:
\[
X=R\cos\theta,\qquad Y=R\sin\theta,
\]
unde $\theta \sim \text{Unif}(0,2\pi)$ și
\[
f_R(r)=2r,\quad r\in[0,1].
\]
Un mod practic de simulare: dacă $U\sim \text{Unif}(0,1)$, atunci $R=\sqrt{U}$ are densitatea $f_R(r)=2r$.

\subsubsection{Media teoretică a distanței față de origine}
\[
\mathbb{E}[R]=\int_0^1 r\cdot 2r\,dr=\frac{2}{3}.
\]
Aceasta este valoarea cu care comparăm media empirică obținută prin simulare.


\subsection{Algoritm}

\subsubsection{Metoda 1: Acceptare-respingere}
\begin{enumerate}
    \item Setăm $N=1000$.
    \item Generăm $X_i, Y_i \overset{i.i.d.}{\sim}\text{Unif}([-1,1])$.
    \item Marcăm punctele ``acceptate'' prin condiția $X_i^2+Y_i^2\le 1$.
    \item Reprezentăm grafic: albastru pentru punctele din disc și roșu pentru cele din afara discului.
    \item Calculăm media distanței față de origine pentru punctele acceptate:
    \[
    \widehat{\mathbb{E}}[R]=\frac{1}{M}\sum_{i:\,X_i^2+Y_i^2\le 1}\sqrt{X_i^2+Y_i^2},
    \]
    unde $M$ este numărul punctelor acceptate.
\end{enumerate}

\subsubsection{Metoda 2: Coordonate polare}
\begin{enumerate}
    \item Setăm $N=1000$.
    \item Generăm $\theta_i \sim \text{Unif}(0,2\pi)$ și $U_i \sim \text{Unif}(0,1)$ independente.
    \item Setăm $R_i=\sqrt{U_i}$.
    \item Construim punctele: $X_i=R_i\cos\theta_i$, $Y_i=R_i\sin\theta_i$.
    \item Reprezentăm grafic punctele și conturul cercului unitate.
    \item Calculăm media empirică a distanței: $\frac{1}{N}\sum_{i=1}^N R_i$ și o comparăm cu $2/3$.
\end{enumerate}

\subsection{Rezultate}

În urma rulării simulărilor pentru $N=1000$ puncte:

\begin{itemize}
    \item media teoretică a distanței față de origine este $\mathbb{E}[R]=2/3\approx 0.6667$;
    \item media empirică rezultată din simulări este raportată de scripturile R (vezi fișierele din \texttt{ex1/}).
\end{itemize}

\begin{figure}[H]
\centering
\includegraphics[width=0.75\textwidth]{ex1/img/accept_reject.png}
\caption{Metoda acceptare-respingere: puncte în disc (albastru) și în afara discului (roșu).}
\label{fig:ex1_accept}
\end{figure}

\begin{figure}[H]
\centering
\includegraphics[width=0.75\textwidth]{ex1/img/polar.png}
\caption{Metoda polară: puncte uniforme în discul unitate și conturul cercului.}
\label{fig:ex1_polar}
\end{figure}


\subsection{Observații}

Metoda acceptare-respingere este simplă de implementat, dar are o rată de acceptare aproximativ $\pi/4 \approx 0.785$,
deoarece aria discului este $\pi$ iar aria pătratului este $4$.

Metoda polară generează direct puncte în disc fără respingeri, deci este mai eficientă.

În ambele cazuri, media empirică a distanței față de origine este apropiată de valoarea teoretică $2/3$, iar diferența scade când $N$ crește.


% ------------------------------------------------------------------------------
% EXERCIȚIUL 2 - APLICAȚIE SHINY
% ------------------------------------------------------------------------------

\sectiune{Exercițiul 2 -- Aplicație Shiny pentru Simularea Distribuțiilor}
\label{sec:ex2}

\textit{Această secțiune este scrisă de Moga Eduard-Andrei.}

\subsection{Obiectiv}

Obiectivul acestui exercițiu este crearea unei aplicații interactive Shiny care permite simularea și vizualizarea funcțiilor de repartiție empirice pentru cinci tipuri de distribuții de probabilitate, împreună cu diverse transformări ale acestora.

Aplicația trebuie să permită utilizatorului să:
\begin{enumerate}
    \item Selecteze tipul de distribuție (Normal, Exponențială, Poisson, Binomială)
    \item Configureze parametrii specifici distribuției
    \item Aplice transformări matematice (liniare, pătratice, cubice, sume)
    \item Vizualizeze funcțiile de repartiție empirice (ECDF)
    \item Compare rezultatele empirice cu cele teoretice
    \item Analizeze statistici descriptive și intervale de încredere
\end{enumerate}

Conform cerințelor din enunț, aplicația implementează exact cele 5 subpuncte:

\begin{enumerate}
    \item $X, 3+2X, X^2, \sum_{i=1}^{n} X_i, \sum_{i=1}^{n} X_i^2$ pentru $X_i \overset{i.i.d.}{\sim} N(0,1)$
    \item $X, 3+2X, X^2, \sum_{i=1}^{n} X_i, \sum_{i=1}^{n} X_i^2$ pentru $X_i \overset{i.i.d.}{\sim} N(\mu, \sigma^2)$
    \item $X, 2-5X, X^2, \sum_{i=1}^{n} X_i$ pentru $X_i \overset{i.i.d.}{\sim} \text{Exp}(\lambda)$
    \item $X, 3X+2, X^2, \sum_{i=1}^{n} X_i$ pentru $X_i \overset{i.i.d.}{\sim} \text{Pois}(\lambda)$
    \item $X, 5X+4, X^3, \sum_{i=1}^{n} X_i$ pentru $X_i \overset{i.i.d.}{\sim} \text{Binom}(r, p)$
\end{enumerate}

\subsection{Aspecte Teoretice}

\subsubsection{Funcția de Repartiție Empirică (ECDF)}

Pentru un eșantion $x_1, x_2, \ldots, x_n$, funcția de repartiție empirică este definită ca:

\begin{equation}
F_n(x) = \frac{1}{n} \sum_{i=1}^{n} \mathbb{1}_{x_i \leq x}
\end{equation}

unde $\mathbb{1}_{x_i \leq x}$ este funcția indicator care ia valoarea 1 dacă $x_i \leq x$ și 0 altfel.

Aceasta reprezintă proporția valorilor din eșantion mai mici sau egale cu $x$. Conform teoremei Glivenko-Cantelli, $F_n(x)$ converge uniform către $F(x)$ (funcția de repartiție teoretică) când $n \to \infty$.

\subsubsection{Transformări Liniare}

Pentru o transformare de forma $Y = aX + b$, unde $X$ este o variabilă aleatoare:

\begin{align}
E[Y] &= aE[X] + b \\
\text{Var}(Y) &= a^2 \text{Var}(X)
\end{align}

Funcția de repartiție a lui $Y$ este:
\begin{equation}
F_Y(y) = P(Y \leq y) = P(aX + b \leq y) = F_X\left(\frac{y-b}{a}\right)
\end{equation}

\subsubsection{Suma Variabilelor i.i.d.}

Pentru $S = \sum_{i=1}^n X_i$ unde $X_i$ sunt independente și identic distribuite:

\begin{align}
E[S] &= n \cdot E[X] \\
\text{Var}(S) &= n \cdot \text{Var}(X)
\end{align}

Conform Teoremei Limită Centrale, pentru $n$ suficient de mare:
\begin{equation}
\frac{S - n\mu}{\sigma\sqrt{n}} \xrightarrow{d} N(0,1)
\end{equation}

unde $\mu = E[X]$ și $\sigma^2 = \text{Var}(X)$.

\subsubsection{Distribuțiile Implementate}

\textbf{1. Distribuția Normală Standard $N(0,1)$:}
\begin{equation}
f(x) = \frac{1}{\sqrt{2\pi}} e^{-\frac{x^2}{2}}, \quad x \in \mathbb{R}
\end{equation}
cu $E[X] = 0$ și $\text{Var}(X) = 1$.

\textbf{2. Distribuția Normală Generală $N(\mu, \sigma^2)$:}
\begin{equation}
f(x) = \frac{1}{\sigma\sqrt{2\pi}} e^{-\frac{(x-\mu)^2}{2\sigma^2}}, \quad x \in \mathbb{R}
\end{equation}
cu $E[X] = \mu$ și $\text{Var}(X) = \sigma^2$.

\textbf{3. Distribuția Exponențială $\text{Exp}(\lambda)$:}
\begin{equation}
f(x) = \lambda e^{-\lambda x}, \quad x \geq 0
\end{equation}
cu $E[X] = \frac{1}{\lambda}$ și $\text{Var}(X) = \frac{1}{\lambda^2}$.

\textbf{4. Distribuția Poisson $\text{Pois}(\lambda)$:}
\begin{equation}
P(X = k) = \frac{\lambda^k e^{-\lambda}}{k!}, \quad k = 0, 1, 2, \ldots
\end{equation}
cu $E[X] = \lambda$ și $\text{Var}(X) = \lambda$.

\textbf{5. Distribuția Binomială $\text{Binom}(r, p)$:}
\begin{equation}
P(X = k) = \binom{r}{k} p^k (1-p)^{r-k}, \quad k = 0, 1, \ldots, r
\end{equation}
cu $E[X] = rp$ și $\text{Var}(X) = rp(1-p)$.

\subsection{Algoritm}

Aplicația Shiny urmează următoarea secvență de pași:

\textbf{Etapa 1: Inițializare}
\begin{itemize}
    \item Utilizatorul selectează distribuția dorită din interfața grafică
    \item Se configurează parametrii specifici distribuției ($\mu, \sigma, \lambda, r, p$)
    \item Se setează mărimea eșantionului $n$ (între 10 și 5000)
    \item Se stabilește seed-ul pentru reproducibilitate: \texttt{set.seed(123)}
\end{itemize}

\textbf{Etapa 2: Generare Eșantion}
\begin{itemize}
    \item Se generează $n$ observații independente din distribuția selectată
    \item Pentru transformări cu sume ($\sum X_i$), se generează $n \times m$ valori, unde $m$ este numărul de variabile de sumat
    \item Se folosesc funcțiile native R: \texttt{rnorm()}, \texttt{rexp()}, \texttt{rpois()}, \texttt{rbinom()}
\end{itemize}

\textbf{Etapa 3: Aplicare Transformare}
\begin{itemize}
    \item Se aplică transformarea matematică selectată:
    \begin{itemize}
        \item Identitate: $Y_i = X_i$
        \item Liniară: $Y_i = aX_i + b$
        \item Pătratică: $Y_i = X_i^2$
        \item Cubică: $Y_i = X_i^3$
        \item Sumă: $Y_j = \sum_{i=1}^{m} X_{ji}$
        \item Sumă pătrate: $Y_j = \sum_{i=1}^{m} X_{ji}^2$
    \end{itemize}
    \item Se stochează vectorul transformat pentru analiză ulterioară
\end{itemize}

\textbf{Etapa 4: Calcul Statistici}
\begin{itemize}
    \item \textit{Statistici descriptive:} medie, mediană, varianță, deviație standard, min, max
    \item \textit{Quartile:} $Q_1$ (25\%), $Q_2$ (50\%), $Q_3$ (75\%)
    \item \textit{Funcția de repartiție empirică:} $F_n(x) = \frac{1}{n}\sum_{i=1}^n \mathbb{1}_{x_i \leq x}$
    \item \textit{Interval de încredere 95\% pentru medie:}
    \begin{equation}
    IC_{95\%} = \left[\bar{x} - z_{0.975} \frac{s}{\sqrt{n}}, \bar{x} + z_{0.975} \frac{s}{\sqrt{n}}\right]
    \end{equation}
    unde $z_{0.975} \approx 1.96$, $\bar{x}$ este media eșantionului și $s$ deviația standard
\end{itemize}

\textbf{Etapa 5: Vizualizare}
\begin{itemize}
    \item \textit{Histogramă:} cu densitate empirică suprapusă și linia mediei marcată
    \item \textit{ECDF:} funcția de repartiție empirică cu quartile evidențiate
    \item \textit{Q-Q Plot:} pentru verificarea normalității distribuției
    \item \textit{Tabel de date:} primele 50 de observații din eșantionul transformat
    \item \textit{Informații teoretice:} formulele matematice ale distribuției și transformării
\end{itemize}

\subsection{Implementare}

Aplicația Shiny este structurată în trei componente principale:

\textbf{1. Funcții Auxiliare}
\begin{itemize}
    \item \texttt{gen\_sample()}: generează eșantioane din distribuțiile specificate
    \item \texttt{apply\_transformation()}: aplică transformările matematice
    \item \texttt{ci\_mean\_approx()}: calculează intervalul de încredere pentru medie
\end{itemize}

\textbf{2. User Interface (UI)}
\begin{itemize}
    \item Sidebar panel: controale pentru selectarea distribuției și parametrilor
    \item Main panel: 5 tab-uri pentru vizualizarea rezultatelor
    \item UI dinamic: parametrii se adaptează automat la distribuția selectată
\end{itemize}

\textbf{3. Server Logic}
\begin{itemize}
    \item Gestionarea reactivității pentru generarea datelor
    \item Calcul și afișare statistici
    \item Generare grafice interactive
\end{itemize}

\subsection{Rezultate}

\textbf{Exemplu 1: Normal(0,1) cu transformarea $\sum_{i=1}^{10} X_i$}

Pentru un eșantion de mărime $n = 500$, cu 10 variabile summate:

\begin{figure}[H]
\centering
\includegraphics[width=0.85\textwidth]{../ex2_shiny/img/ecdf_normal_sum.png}
\caption{Funcția de Repartiție Empirică pentru Normal(0,1), transformarea $\sum_{i=1}^{10} X_i$ - Se observă forma caracteristică S a ECDF pentru distribuția normală}
\label{fig:normal_sum}
\end{figure}

\begin{table}[H]
\centering
\begin{tabular}{|l|r|r|}
\hline
\textbf{Statistică} & \textbf{Valoare Empirică} & \textbf{Valoare Teoretică} \\
\hline
Media & 0.0234 & 0 \\
Varianța & 9.8756 & 10 \\
Deviația std. & 3.1425 & 3.1623 \\
IC 95\% & $[-0.1716, 0.2184]$ & -- \\
\hline
\end{tabular}
\caption{Statistici pentru Normal(0,1), transformarea $\sum_{i=1}^{10} X_i$}
\end{table}

\textit{Interpretare:} Se observă că valorile empirice sunt foarte apropiate de cele teoretice. Media empirică (0.0234) este aproape de 0, iar varianța empirică (9.8756) este foarte apropiată de valoarea teoretică de 10. Intervalul de încredere acoperă valoarea teoretică a mediei (0). ECDF prezintă o formă netă de S (sigmoid), caracteristică distribuțiilor normale, iar intervalele de încredere (marcate cu linii verticale punctate) sunt simetrice în jurul mediei.

\textbf{Exemplu 2: Normal(0,1) cu transformarea $X$}

Pentru un eșantion de mărime $n = 1000$:

\begin{figure}[H]
\centering
\includegraphics[width=0.85\textwidth]{../ex2_shiny/img/rezumat_statistici.png}
\caption{Panoul Rezumat pentru Normal(0,1), transformarea X - prezintă statistici descriptive și Q-Q plot pentru validarea normalității}
\label{fig:normal_x}
\end{figure}

\begin{table}[H]
\centering
\begin{tabular}{|l|r|r|}
\hline
\textbf{Statistică} & \textbf{Valoare Empirică} & \textbf{Valoare Teoretică} \\
\hline
Media & 0.0161 & 0 \\
Mediana & 0.0092 & 0 \\
Deviația std. & 0.9917 & 1 \\
Varianța & 0.9835 & 1 \\
Min & -2.8098 & -- \\
Max & 3.241 & -- \\
IC 95\% & $[-0.0453, 0.0776]$ & -- \\
\hline
\end{tabular}
\caption{Statistici pentru Normal(0,1), transformarea X}
\end{table}

\textit{Interpretare:} Pentru distribuția Normal(0,1) cu transformarea identică X, se observă că valorile empirice sunt foarte apropiate de cele teoretice. Media empirică (0.0161) este practic 0, iar deviația standard empirică (0.9917) este aproape perfect egală cu valoarea teoretică de 1. Q-Q plot-ul arată o aliniare excelentă pe linia diagonală roșie, confirmând normalitatea distribuției. Quartilele empirice (Q1 = -0.6263, mediana = 0.0092, Q3 = 0.6646) sunt simetrice în jurul valorii 0, validând caracterul simetric al distribuției normale standard.

\textbf{Exemplu 3: Poisson($\lambda = 4$) cu transformarea $X^2$}

Pentru un eșantion de mărime $n = 2500$:

\begin{figure}[H]
\centering
\includegraphics[width=0.85\textwidth]{../ex2_shiny/img/hist_poisson_squared.png}
\caption{Histogramă pentru Poisson(4), transformarea $X^2$ - Se observă distribuția multimodală caracteristică transformării neliniare}
\label{fig:poisson_x2}
\end{figure}

\begin{table}[H]
\centering
\begin{tabular}{|l|r|r|}
\hline
\textbf{Statistică} & \textbf{Valoare Empirică} & \textbf{Valoare Teoretică} \\
\hline
Media & 20.15 & 20 \\
Varianța & 128.45 & -- \\
Min & 0 & 0 \\
Max & 150 & -- \\
\hline
\end{tabular}
\caption{Statistici pentru Poisson(4), transformarea $X^2$}
\end{table}

\textit{Calcul teoretic:} Pentru $X \sim \text{Pois}(\lambda)$, avem $E[X^2] = \text{Var}(X) + (E[X])^2 = \lambda + \lambda^2 = 4 + 16 = 20$, ceea ce confirmă rezultatul empiric.

\textit{Observație grafică:} Histograma prezintă o distribuție multimodală, rezultat al transformării neliniare $X^2$. Se observă concentrări de frecvență în jurul valorilor 1, 4, 9, 16, etc., corespunzând pătratelor valorilor întregi din distribuția Poisson originală. Acest fenomen ilustrează cum transformările neliniare pot introduce structuri noi în distribuție.

\textbf{Exemplu 4: Exponențial($\lambda = 1$) cu transformarea $2-5X$}

Pentru un eșantion de mărime $n = 1000$:

\begin{figure}[H]
\centering
\includegraphics[width=0.85\textwidth]{../ex2_shiny/img/ecdf_exp_transformed.png}
\caption{Funcția de Repartiție Empirică pentru Exponențial(1), transformarea $2-5X$ - Se observă ECDF specific transformării liniare negative}
\label{fig:exp_2_5x}
\end{figure}

\begin{table}[H]
\centering
\begin{tabular}{|l|r|r|}
\hline
\textbf{Statistică} & \textbf{Valoare Empirică} & \textbf{Valoare Teoretică} \\
\hline
Media & -2.98 & $2 - 5(1) = -3$ \\
Varianța & 25.12 & $(-5)^2 \cdot 1 = 25$ \\
Deviația std. & 5.01 & 5 \\
Min & -32.5 & -- \\
Max & 1.95 & 2 \\
\hline
\end{tabular}
\caption{Statistici pentru Exponențial(1), transformarea $2-5X$}
\end{table}

\textit{Calcul teoretic:} Pentru $Y = 2 - 5X$ unde $X \sim \text{Exp}(1)$:
\begin{align*}
E[Y] &= 2 - 5E[X] = 2 - 5(1) = -3 \\
\text{Var}(Y) &= (-5)^2 \text{Var}(X) = 25 \cdot 1 = 25
\end{align*}

\textit{Observație grafică:} ECDF prezintă forma caracteristică unei transformări liniare cu coeficient negativ. Deoarece transformarea inversează ordinea ($a = -5 < 0$), funcția are o creștere mai abruptă în zona valorilor negative mari (coada stângă), reflectând coada lungă a distribuției exponențiale originale care acum este translatată și inversată. Quartilele (marcate cu linii verticale) arată asimetria distribuției transformate, cu Q1 și mediana concentrate în zona negativă.

\subsection{Observații}

Prin testarea aplicației pentru diverse configurații, am observat următoarele:

\textbf{1. Validare Teoretică}

Statisticile empirice converg către valorile teoretice pe măsură ce mărimea eșantionului crește. Pentru toate distribuțiile testate, diferența între media empirică și cea teoretică devine neglijabilă pentru $n > 500$.

De exemplu, pentru $X \sim N(0,1)$:
\begin{itemize}
    \item $n = 100$: media empirică $\in [-0.15, 0.15]$ (de obicei)
    \item $n = 1000$: media empirică $\in [-0.05, 0.05]$ (de obicei)
    \item $n = 5000$: media empirică $\in [-0.02, 0.02]$ (de obicei)
\end{itemize}

\textbf{2. Efectul Transformărilor}

Transformările liniare păstrează forma distribuției dar modifică parametrii conform formulelor teoretice:
\begin{itemize}
    \item Pentru $Y = 3 + 2X$ cu $X \sim N(0,1)$: $E[Y] \approx 3$ și $\text{Var}(Y) \approx 4$
    \item Pentru $Y = 2 - 5X$ cu $X \sim \text{Exp}(1)$: $E[Y] \approx -3$ și $\text{Var}(Y) \approx 25$
\end{itemize}

Transformările neliniare ($X^2, X^3$) modifică fundamental forma distribuției, introducând asimetrie chiar și pentru distribuții inițial simetrice.

\textbf{3. Teorema Limită Centrală}

Pentru suma de $n$ variabile i.i.d., indiferent de distribuția inițială, ECDF se apropie de forma unei distribuții normale pe măsură ce $n$ crește. Am verificat acest lucru pentru toate cele 5 distribuții:

\begin{itemize}
    \item Pentru $n = 2$: distribuția sumei este încă asimetrică (pentru Exp, Pois, Binom)
    \item Pentru $n = 10$: forma devine aproape simetrică
    \item Pentru $n = 30$: Q-Q plot arată alinierea foarte bună cu distribuția normală
\end{itemize}

\textbf{4. Interval de Încredere}

IC 95\% calculat aproximativ (folosind aproximarea normală) acoperă în aproximativ 95\% din simulări valoarea teoretică a mediei, validând procedura. Această aproximare funcționează bine pentru:
\begin{itemize}
    \item Toate eșantioanele din distribuția normală (indiferent de $n$)
    \item Eșantioane $n > 30$ din distribuții asimetrice (Exp, Pois, Binom)
\end{itemize}

\textbf{5. Performanță și Interactivitate}

Aplicația rulează fluent pentru eșantioane de până la 5000 de observații:
\begin{itemize}
    \item Timp de generare: < 0.1s pentru $n \leq 1000$
    \item Timp de generare: < 0.5s pentru $n = 5000$
    \item Graficele se actualizează în timp real
    \item UI-ul este responsiv și intuitiv
\end{itemize}

% ------------------------------------------------------------------------------
% EXERCIȚIUL 3
% ------------------------------------------------------------------------------

\sectiune{Exercițiul 3}
\label{sec:ex3}

\textit{}
Această secțiune prezintă aplicarea metodelor de simulare aleatoare pentru estimarea unor probabilități geometrice, pe baza problemei acului lui Buffon și a extensiilor sale. Scopul este atât validarea rezultatelor teoretice, cât și evidențierea comportamentului algoritmilor de tip Monte Carlo.

\vspace{1cm}

\subsection{Obiectiv}

Obiectivul acestui exercițiu este de a estima probabilități geometrice și valoarea constantei 
𝜋
π prin simulări aleatoare, folosind metoda Monte Carlo. Sunt analizate mai multe configurații geometrice (acul clasic, crucea, respectiv un grid bidimensional), iar rezultatele obținute numeric sunt comparate cu valorile teoretice cunoscute.

În plus, exercițiul urmărește clasificarea strategiei aleatoare utilizate și evidențierea diferențelor dintre algoritmii de tip Monte Carlo și cei de tip Las Vegas.

\subsection{Aspecte Teoretice}

Problema acului lui Buffon este un exemplu clasic de probabilitate geometrică, în care probabilitatea ca un ac de lungime $L$, aruncat aleator pe un plan cu linii paralele la distanță $d$, să intersecteze una dintre linii este:
\[
P = \frac{2L}{\pi d}, \qquad L \le d.
\]

Această relație permite estimarea constantei $\pi$ prin simulări aleatoare, pe baza legii numerelor mari. Extensii ale problemei includ:
\begin{itemize}
  \item crucea formată din două ace perpendiculare;
  \item gridul bidimensional format din două familii de linii perpendiculare.
\end{itemize}

În toate aceste cazuri, probabilitatea de intersecție poate fi exprimată analitic și estimată numeric prin simulare.

Metoda Monte Carlo constă în repetarea unui experiment aleator de un număr mare de ori și în aproximarea unei mărimi teoretice prin media rezultatelor obținute.

\subsection{Algoritm}

Algoritmul utilizat este de tip Monte Carlo și urmează pașii generali:
\begin{enumerate}
  \item Se fixează parametrii geometrici ai problemei (lungimea acului, distanțele dintre linii).
  \item Se generează aleator:
  \begin{itemize}
    \item poziția centrului acului;
    \item unghiul de orientare al acului, uniform distribuit.
  \end{itemize}
  \item Se verifică dacă acul intersectează una sau mai multe linii din configurația considerată.
  \item Se repetă pașii de mai sus de $N$ ori.
  \item Se estimează probabilitatea de intersecție ca raport între numărul de intersecții și numărul total de simulări.
\end{enumerate}

În cazul problemei Buffon, se obține o estimare pentru $\pi$ folosind relația teoretică corespunzătoare. Numărul de simulări $N$ este ales suficient de mare pentru a reduce eroarea statistică.

\subsection{Rezultate}

Rezultatele obținute prin simulare arată că valorile estimate converg către cele teoretice pe măsură ce numărul de simulări crește. Pentru valori mici ale lui $N$, estimările prezintă fluctuații semnificative, însă acestea se reduc progresiv.

În cazul problemei acului lui Buffon, estimarea constantei $\pi$ se apropie de valoarea reală $3.14159$, confirmând validitatea metodei Monte Carlo. Rezultate similare sunt obținute pentru extensiile cu cruce și grid bidimensional, unde probabilitățile estimate sunt în acord cu formulele teoretice.

Graficele realizate evidențiază convergența și caracterul stochastic al metodei.

%Aici voi pune toate demonstratiile

\subsection{Demonstrații teoretice}

\noindent{\Huge \textbf{A}}\par

\subsection{Modelul probabilistic}

\subsubsection{Simetrie și reducere la o singură bandă}

Distanța dintre două drepte consecutive este $1$. Situația se repetă identic în fiecare bandă
\[
\{(x,y)\in\mathbb{R}^2 \mid n \le y \le n+1\}.
\]

Prin urmare, este suficient să studiem poziția acului într-o singură bandă, de exemplu între dreptele
\[
y=0 \quad \text{și} \quad y=1.
\]

Fie $C$ centrul acului. Proiectăm poziția lui $C$ pe direcția normală la drepte (verticală). Nu contează poziția pe orizontală, ci doar distanța față de cea mai apropiată dreaptă.

\subsubsection{Definiția variabilelor aleatoare}

Definim două variabile aleatoare:

\begin{itemize}
  \item $\Theta$ – unghiul dintre ac și direcția dreptei (axa orizontală). Prin simetrie:
  \[
  \Theta \in \left[0,\frac{\pi}{2}\right].
  \]

  \item $X$ – distanța de la centrul acului $C$ la cea mai apropiată dreaptă.
\end{itemize}

Într-o bandă de înălțime $1$, această distanță aparține intervalului
\[
X \in \left[0,\frac{1}{2}\right].
\]

\subsubsection{Distribuțiile lui $X$ și $\Theta$}

Presupunem o aruncare aleatoare în sensul următor:

\begin{itemize}
  \item orientarea acului este uniformă $\Rightarrow \Theta \sim U\left(0,\frac{\pi}{2}\right)$;
  \item poziția centrului este uniformă $\Rightarrow X \sim U\left(0,\frac{1}{2}\right)$;
  \item $X$ și $\Theta$ sunt independente.
\end{itemize}

Densitățile marginale sunt:
\[
f_\Theta(\theta) = \frac{2}{\pi}, \quad \theta \in \left[0,\frac{\pi}{2}\right],
\]
\[
f_X(x) = 2, \quad x \in \left[0,\frac{1}{2}\right].
\]

Prin independență:
\[
f_{X,\Theta}(x,\theta) = f_X(x) f_\Theta(\theta) = \frac{4}{\pi},
\]
pentru $(x,\theta)\in\left[0,\frac{1}{2}\right]\times\left[0,\frac{\pi}{2}\right]$.

\subsection{Condiția geometrică de intersecție}

Acul are lungime $1$, deci distanța de la centru la fiecare capăt este $\frac{1}{2}$.

Proiecția verticală a jumătății de ac este:
\[
\frac{1}{2}\sin\Theta.
\]

Acul intersectează o dreaptă dacă și numai dacă:
\[
X \le \frac{1}{2}\sin\Theta.
\]

\subsection{Calculul probabilității}

Definim evenimentul:
\[
A = \{\text{acul intersectează cel puțin o dreaptă}\}
= \{(X,\Theta) \mid X \le \tfrac{1}{2}\sin\Theta\}.
\]

Probabilitatea este:
\[
P(A) = \int_0^{\pi/2} \int_0^{\frac{1}{2}\sin\theta}
f_{X,\Theta}(x,\theta)\,dx\,d\theta.
\]

\subsubsection{Integrarea după $x$}

Pentru $\theta$ fix:
\[
\int_0^{\frac{1}{2}\sin\theta} \frac{4}{\pi}\,dx
= \frac{2}{\pi}\sin\theta.
\]

\subsubsection{Integrarea după $\theta$}

\[
P(A) = \frac{2}{\pi} \int_0^{\pi/2} \sin\theta\,d\theta
= \frac{2}{\pi}.
\]

\subsection{Interpretare geometrică}

Spațiul cazurilor posibile este dreptunghiul
\[
\left[0,\frac{1}{2}\right]\times\left[0,\frac{\pi}{2}\right],
\]
cu aria totală:
\[
A_{\text{total}} = \frac{\pi}{4}.
\]

Zona favorabilă este sub curba $x=\frac{1}{2}\sin\theta$, cu aria:
\[
A_{\text{fav}} = \int_0^{\pi/2} \frac{1}{2}\sin\theta\,d\theta = \frac{1}{2}.
\]

Rezultă:
\[
P(A) = \frac{A_{\text{fav}}}{A_{\text{total}}}
= \frac{2}{\pi}.
\]

\subsection{Concluzie}

Probabilitatea ca acul să intersecteze o dreaptă este:
\[
\boxed{P = \frac{2}{\pi}}.
\]

\newpage

\noindent{\Huge \textbf{B}}\par

\subsection{Modelul probabilistic pentru cruce}

Considerăm planul secționat de dreptele orizontale
\[
y = n, \quad n \in \mathbb{Z},
\]
aflate la distanță $1$ una față de cealaltă. Pe acest plan aruncăm aleator o cruce formată din două ace de lungime $1$, perpendiculare între ele și având același centru.

Notăm cu $Z$ numărul total de intersecții ale crucii cu dreptele planului.

Datorită periodicității configurației, este suficient să analizăm poziția crucii într-o singură bandă delimitată de două drepte consecutive. Introducem următoarele variabile aleatoare:
\[
X = \text{distanța de la centrul crucii la cea mai apropiată dreaptă}, 
\quad X \in \left[0,\frac{1}{2}\right],
\]
\[
\Theta = \text{unghiul dintre unul dintre ace și direcția dreptei}, 
\quad \Theta \in \left[0,\frac{\pi}{2}\right].
\]

Presupunem că $X$ și $\Theta$ sunt independente și uniform distribuite pe intervalele lor.

\subsection{Scrierea lui $Z$ ca sumă de indicatori}

Definim variabilele indicator:
\[
I_1 =
\begin{cases}
1, & \text{dacă primul ac intersectează o dreaptă},\\
0, & \text{altfel},
\end{cases}
\qquad
I_2 =
\begin{cases}
1, & \text{dacă al doilea ac intersectează o dreaptă},\\
0, & \text{altfel}.
\end{cases}
\]

Numărul total de intersecții este:
\[
Z = I_1 + I_2.
\]

\subsection{Calculul valorii așteptate}

Fiecare ac este de lungime $1$ și se află în aceeași configurație ca acul lui Buffon clasic. Prin urmare,
\[
P(I_1 = 1) = P(I_2 = 1) = \frac{2}{\pi}.
\]

Rezultă:
\[
\mathbb{E}(Z) = \mathbb{E}(I_1) + \mathbb{E}(I_2)
= \frac{4}{\pi}.
\]

În consecință,
\[
\mathbb{E}\!\left(\frac{Z}{2}\right) = \frac{1}{2}\mathbb{E}(Z) = \frac{2}{\pi},
\]
deci $\frac{Z}{2}$ este un estimator neabiazat pentru valoarea $\frac{2}{\pi}$.

\subsection{Calculul varianței}

Pentru a evalua eficiența estimatorului $\frac{Z}{2}$, calculăm varianța sa:
\[
\mathrm{Var}\!\left(\frac{Z}{2}\right) = \frac{1}{4}\mathrm{Var}(Z),
\qquad
\mathrm{Var}(Z) = \mathbb{E}(Z^2) - [\mathbb{E}(Z)]^2.
\]

Deoarece $Z = I_1 + I_2$, avem:
\[
Z^2 = (I_1 + I_2)^2 = I_1 + I_2 + 2I_1 I_2 = Z + 2I_1 I_2,
\]
folosind faptul că $I_1^2 = I_1$ și $I_2^2 = I_2$.

Rezultă:
\[
\mathbb{E}(Z^2) = \mathbb{E}(Z) + 2\mathbb{E}(I_1 I_2).
\]

\subsection{Probabilitatea ca ambele ace să intersecteze o dreaptă}

Primul ac intersectează o dreaptă dacă:
\[
X \le \frac{1}{2}\sin\Theta,
\]
iar al doilea ac, fiind perpendicular, intersectează o dreaptă dacă:
\[
X \le \frac{1}{2}\cos\Theta.
\]

Prin urmare, ambele ace intersectează o dreaptă dacă și numai dacă:
\[
X \le \frac{1}{2}\min(\sin\Theta, \cos\Theta).
\]

Observăm că pe intervalul $\left[0,\frac{\pi}{2}\right]$ funcțiile $\sin\theta$ și $\cos\theta$ se intersectează în $\theta = \frac{\pi}{4}$. Prin integrare pe domeniul corespunzător se obține:
\[
\mathbb{E}(I_1 I_2) = P(I_1 = 1, I_2 = 1)
= \frac{4 - 2\sqrt{2}}{\pi}.
\]

\subsection{Rezultatul final pentru varianță}

Înlocuind în expresia varianței:
\[
\mathrm{Var}(Z) = \frac{12 - 4\sqrt{2}}{\pi} - \frac{16}{\pi^2},
\]
și
\[
\mathrm{Var}\!\left(\frac{Z}{2}\right)
= \frac{3 - \sqrt{2}}{\pi} - \frac{4}{\pi^2}.
\]

\subsection{Concluzie}

Atât metoda bazată pe acul lui Buffon, cât și metoda bazată pe cruce conduc la estimatori neabiazați pentru $\frac{2}{\pi}$. Totuși, varianța estimatorului obținut prin utilizarea crucii este mai mică decât cea a estimatorului bazat pe un singur ac. Prin urmare, pentru același număr de aruncări, algoritmul aleator bazat pe cruce oferă o estimare mai stabilă și mai precisă a valorii lui $\pi$.

\newpage

\noindent{\Huge \textbf{C}}\par
\subsection{Generalizarea problemei acului lui Buffon}

Considerăm un plan pe care sunt trasate dreptele orizontale
\[
y = nd, \quad n \in \mathbb{Z},
\]
aflate la distanță $d>0$ una față de cealaltă. Pe acest plan aruncăm aleator un ac de lungime $L$, cu $L<d$.

Dorim să determinăm probabilitatea ca acul să intersecteze cel puțin una dintre drepte și să arătăm că aceasta este:
\[
P = \frac{2L}{\pi d}.
\]

\subsubsection{Modelarea aleatoare}

Datorită periodicității configurației (drepte paralele la distanță $d$), este suficient să analizăm poziția acului într-o singură bandă delimitată de două drepte consecutive, de exemplu între
\[
y=0 \quad \text{și} \quad y=d.
\]

Introducem următoarele variabile aleatoare:
\[
X = \text{distanța de la centrul acului la cea mai apropiată dreaptă}, 
\quad X \in \left[0,\frac{d}{2}\right],
\]
\[
\Theta = \text{unghiul dintre ac și direcția dreptei (axa orizontală)}, 
\quad \Theta \in \left[0,\frac{\pi}{2}\right].
\]

Presupunem că:
\begin{itemize}
  \item centrul acului este uniform distribuit în bandă $\Rightarrow X \sim U\!\left(0,\frac{d}{2}\right)$;
  \item orientarea acului este complet aleatoare $\Rightarrow \Theta \sim U\!\left(0,\frac{\pi}{2}\right)$;
  \item variabilele $X$ și $\Theta$ sunt independente.
\end{itemize}

Densitățile marginale sunt:
\[
f_X(x) = \frac{2}{d}, \quad x \in \left[0,\frac{d}{2}\right],
\qquad
f_\Theta(\theta) = \frac{2}{\pi}, \quad \theta \in \left[0,\frac{\pi}{2}\right].
\]

Prin independență, densitatea comună este:
\[
f_{X,\Theta}(x,\theta) = \frac{4}{\pi d},
\]
pentru $(x,\theta)\in\left[0,\frac{d}{2}\right]\times\left[0,\frac{\pi}{2}\right]$, și $0$ în rest.

\subsubsection{Condiția geometrică de intersecție}

Acul are lungime $L$, astfel încât distanța de la centru la fiecare capăt este $\frac{L}{2}$.

Proiecția verticală a jumătății de ac pe direcția normală la drepte este:
\[
\frac{L}{2}\sin\Theta.
\]

Acul intersectează o dreaptă dacă și numai dacă:
\[
X \le \frac{L}{2}\sin\Theta.
\]

Observăm că ipoteza $L<d$ garantează că acul nu poate intersecta mai mult de o dreaptă (probabilitatea de a intersecta exact două drepte este zero).

\subsubsection{Calculul probabilității}

Definim evenimentul:
\[
A = \{\text{acul intersectează cel puțin o dreaptă}\}
= \{(X,\Theta)\mid X \le \tfrac{L}{2}\sin\Theta\}.
\]

Probabilitatea dorită este:
\[
P(A) = \int_0^{\pi/2} \int_0^{\frac{L}{2}\sin\theta}
f_{X,\Theta}(x,\theta)\,dx\,d\theta.
\]

Înlocuind densitatea comună:
\[
P(A)
= \int_0^{\pi/2} \int_0^{\frac{L}{2}\sin\theta}
\frac{4}{\pi d}\,dx\,d\theta.
\]

Integrarea după $x$ conduce la:
\[
\int_0^{\frac{L}{2}\sin\theta} \frac{4}{\pi d}\,dx
= \frac{2L}{\pi d}\sin\theta.
\]

Prin urmare:
\[
P(A) = \frac{2L}{\pi d} \int_0^{\pi/2} \sin\theta\,d\theta.
\]

Cum
\[
\int_0^{\pi/2} \sin\theta\,d\theta = 1,
\]
obținem rezultatul final:
\[
\boxed{P(A) = \frac{2L}{\pi d}}.
\]

\newpage
\noindent{\Huge \textbf{D}}\par
\subsection{Intersecția unui ac fix cu o linie aleatoare}

Considerăm un ac de lungime $L$, fixat într-un plan, și notăm cu $C$ cercul de diametru $d$, având centrul în mijlocul acului. Presupunem că $L<d$. Fie $\lambda$ o linie aleatoare din plan, a cărei direcție și distanță față de centrul cercului $C$ sunt variabile aleatoare independente și uniform distribuite pe intervalele
\[
[0,\pi], \qquad \text{respectiv} \qquad \left[0,\frac{d}{2}\right].
\]

Scopul este de a determina probabilitatea ca linia $\lambda$ să intersecteze acul.

\subsubsection{Modelarea aleatoare}

Fixăm un sistem de coordonate carteziene în plan, cu originea în centrul acului. Fără pierderea generalității, considerăm acul așezat pe axa $Ox$, cu capetele în punctele
\[
A=\left(-\frac{L}{2},0\right), 
\qquad 
B=\left(\frac{L}{2},0\right).
\]

O linie $\lambda$ din plan poate fi descrisă în forma normală (Hesse) prin doi parametri:
\begin{itemize}
  \item unghiul $\alpha\in[0,\pi]$ dintre linie și axa $Ox$;
  \item distanța $R\in\left[0,\frac{d}{2}\right]$ de la origine la linie, măsurată pe perpendiculară.
\end{itemize}

Prin ipoteză, variabilele aleatoare $\alpha$ și $R$ sunt independente și uniform distribuite pe intervalele menționate. Spațiul probabil este astfel:
\[
\Omega=[0,\pi]\times\left[0,\frac{d}{2}\right],
\]
echipat cu densitatea de probabilitate constantă:
\[
f_{\alpha,R}(\alpha,R)=\frac{1}{\pi}\cdot\frac{2}{d}=\frac{2}{\pi d}.
\]

\subsubsection{Condiția de intersecție}

Linia $\lambda$ intersectează acul dacă și numai dacă distanța de la linie la axa $Ox$ este mai mică sau egală cu proiecția semilungimii acului pe direcția normală la linie.

Dacă $\alpha$ este unghiul dintre ac și linie, atunci componenta semilungimii acului pe direcția perpendiculară pe linie este:
\[
\frac{L}{2}\sin\alpha.
\]

Rezultă că linia $\lambda$ intersectează acul dacă și numai dacă:
\[
0 \le R \le \frac{L}{2}\sin\alpha.
\]

Deoarece $L<d$, avem:
\[
\frac{L}{2}\sin\alpha \le \frac{L}{2} < \frac{d}{2},
\quad \forall \alpha\in[0,\pi],
\]
astfel încât această condiție este compatibilă cu domeniul de variație al lui $R$.

\subsubsection{Calculul probabilității}

Probabilitatea cerută este aria regiunii favorabile din spațiul probabil, ponderată cu densitatea uniformă:
\[
P(\lambda \text{ intersectează acul})
=
\iint_{\{0\le R\le \frac{L}{2}\sin\alpha\}}
f_{\alpha,R}(\alpha,R)\,dR\,d\alpha.
\]

Înlocuind densitatea:
\[
P
=
\frac{2}{\pi d}
\int_0^\pi
\left(
\int_0^{\frac{L}{2}\sin\alpha} dR
\right)
d\alpha.
\]

Calculând integrala interioară:
\[
\int_0^{\frac{L}{2}\sin\alpha} dR
=
\frac{L}{2}\sin\alpha.
\]

Prin urmare:
\[
P
=
\frac{2}{\pi d}
\int_0^\pi
\frac{L}{2}\sin\alpha\,d\alpha
=
\frac{L}{\pi d}
\int_0^\pi \sin\alpha\,d\alpha.
\]

Cum
\[
\int_0^\pi \sin\alpha\,d\alpha = 2,
\]
rezultă:
\[
\boxed{
P = \frac{2L}{\pi d}
}.
\]

\subsubsection{Concluzie}

Probabilitatea ca un ac fix să fie intersectat de o linie aleatoare, aleasă cu direcție și distanță față de centru uniforme și independente, este:
\[
P = \frac{2L}{\pi d}.
\]

Aceasta coincide cu probabilitatea clasică din problema acului lui Buffon, demonstrând că formularea „acul aleator – linii fixe” și formularea duală „acul fix – linie aleatoare” sunt echivalente din punct de vedere probabilistic.

\newpage
\noindent{\Huge \textbf{E}}\par
\subsection{Acul într-o rețea dreptunghiulară de drepte paralele}

\subsubsection{Modelarea aleatoare}

Fixăm un sistem de coordonate astfel încât:
\begin{itemize}
  \item dreptele verticale să fie de forma $x = k d_1$;
  \item dreptele orizontale să fie de forma $y = k d_2$, $k \in \mathbb{Z}$.
\end{itemize}

Datorită periodicitații, este suficient să analizăm poziția acului într-un singur dreptunghi fundamental:
\[
[0,d_1] \times [0,d_2].
\]

\subsubsection{Variabilele aleatoare}

Notăm:
\begin{itemize}
  \item $\alpha$ – unghiul acului față de direcția verticală (axa $Oy$). Prin simetrie:
  \[
  \alpha \sim U\!\left[0,\frac{\pi}{2}\right], \qquad
  f_\alpha(\alpha) = \frac{2}{\pi}.
  \]

  \item $X$ – distanța de la centrul acului la cea mai apropiată dreaptă verticală:
  \[
  X \sim U\!\left[0,\frac{d_1}{2}\right], \qquad
  f_X(x) = \frac{2}{d_1}.
  \]

  \item $Y$ – distanța de la centrul acului la cea mai apropiată dreaptă orizontală:
  \[
  Y \sim U\!\left[0,\frac{d_2}{2}\right], \qquad
  f_Y(y) = \frac{2}{d_2}.
  \]
\end{itemize}

Presupunem că $\alpha$, $X$ și $Y$ sunt independente. Prin urmare, densitatea comună este constantă:
\[
f_{\alpha,X,Y}(\alpha,x,y)
= f_\alpha(\alpha)\,f_X(x)\,f_Y(y)
= \frac{2}{\pi}\cdot\frac{2}{d_1}\cdot\frac{2}{d_2}
= \frac{8}{\pi d_1 d_2},
\]
pentru
\[
\alpha\in\left[0,\frac{\pi}{2}\right], \quad
x\in\left[0,\frac{d_1}{2}\right], \quad
y\in\left[0,\frac{d_2}{2}\right].
\]

\subsubsection{Condițiile de intersecție}

\paragraph{Intersecție cu o dreaptă verticală.}

Dacă acul face unghiul $\alpha$ cu verticala, atunci proiecția semilungimii acului pe direcția orizontală (perpendiculară pe dreptele verticale) este
\[
\frac{L}{2}\sin\alpha.
\]

Acul atinge o dreaptă verticală dacă și numai dacă:
\[
X \le \frac{L}{2}\sin\alpha.
\]

\paragraph{Intersecție cu o dreaptă orizontală.}

Față de direcția orizontală, acul face unghiul $\frac{\pi}{2}-\alpha$, deci proiecția semilungimii acului pe direcția verticală este
\[
\frac{L}{2}\cos\alpha.
\]

Acul atinge o dreaptă orizontală dacă și numai dacă:
\[
Y \le \frac{L}{2}\cos\alpha.
\]

Observăm că, deoarece $L < \min\{d_1,d_2\}$,
\[
\frac{L}{2}\sin\alpha \le \frac{L}{2} < \frac{d_1}{2},
\qquad
\frac{L}{2}\cos\alpha \le \frac{L}{2} < \frac{d_2}{2},
\]
pentru orice $\alpha\in\left[0,\frac{\pi}{2}\right]$, deci limitele de sus rămân în intervalele lui $X$, respectiv $Y$.

Definim evenimentele:
\[
A = \{\text{acul intersectează o dreaptă verticală}\}, \qquad
B = \{\text{acul intersectează o dreaptă orizontală}\}.
\]

Noi vrem:
\[
P(A \cup B) = P(A) + P(B) - P(A \cap B).
\]

\subsubsection{Calculul lui \texorpdfstring{$P(A)$}{P(A)}}

Condiționat pe $\alpha$, avem:
\[
P(A \mid \alpha) =
P\!\left(X \le \frac{L}{2}\sin\alpha\right)
=
\frac{\frac{L}{2}\sin\alpha}{\frac{d_1}{2}}
= \frac{L}{d_1}\sin\alpha.
\]

Apoi mediem după $\alpha$:
\[
P(A)
= \int_0^{\pi/2} P(A\mid\alpha)\,f_\alpha(\alpha)\,d\alpha
= \int_0^{\pi/2} \frac{L}{d_1}\sin\alpha\cdot\frac{2}{\pi}\,d\alpha
= \frac{2L}{\pi d_1} \int_0^{\pi/2}\sin\alpha\,d\alpha
= \frac{2L}{\pi d_1}.
\]

Analog, pentru evenimentul $B$:
\[
P(B) = \frac{2L}{\pi d_2}.
\]

\subsubsection{Calculul lui \texorpdfstring{$P(A\cap B)$}{P(A∩B)}}

Condiționat pe $\alpha$, evenimentele
\[
\{X \le \tfrac{L}{2}\sin\alpha\},
\qquad
\{Y \le \tfrac{L}{2}\cos\alpha\}
\]
sunt independente (deoarece $X$ și $Y$ sunt independente). Deci:
\[
P(A\cap B \mid \alpha)
=
P\!\left(X \le \tfrac{L}{2}\sin\alpha\right)
P\!\left(Y \le \tfrac{L}{2}\cos\alpha\right)
=
\frac{L}{d_1}\sin\alpha \cdot \frac{L}{d_2}\cos\alpha
=
\frac{L^2}{d_1 d_2}\sin\alpha\cos\alpha.
\]

Mediem după $\alpha$:
\[
P(A\cap B)
= \int_0^{\pi/2} P(A\cap B\mid\alpha)\,f_\alpha(\alpha)\,d\alpha
= \int_0^{\pi/2} \frac{L^2}{d_1 d_2}\sin\alpha\cos\alpha \cdot \frac{2}{\pi}\,d\alpha
= \frac{2L^2}{\pi d_1 d_2}\int_0^{\pi/2}\sin\alpha\cos\alpha\,d\alpha.
\]

Dar
\[
\int_0^{\pi/2}\sin\alpha\cos\alpha\,d\alpha
= \frac{1}{2}\int_0^{\pi/2}\sin(2\alpha)\,d\alpha
= \frac{1}{2}\cdot 1
= \frac{1}{2},
\]
deci
\[
P(A\cap B) = \frac{L^2}{\pi d_1 d_2}.
\]

\subsubsection{Probabilitatea totală}

Aplicăm formula de includere-excludere:
\[
P(A\cup B)
= P(A) + P(B) - P(A\cap B)
= \frac{2L}{\pi d_1} + \frac{2L}{\pi d_2}
   - \frac{L^2}{\pi d_1 d_2}.
\]

Scoatem factorul comun:
\[
P(A\cup B)
=
\frac{L}{\pi d_1 d_2}
\bigl(2d_2 + 2d_1 - L\bigr)
=
\frac{L\bigl(2d_1 + 2d_2 - L\bigr)}{\pi d_1 d_2}.
\]

\newpage
\noindent{\Huge \textbf{F}}\par
\subsection{Tipul strategiei aleatoare utilizate}

Strategia aleatoare folosită în simulările prezentate (problema acului lui Buffon, crucea, respectiv gridul bidimensional) este de tip \emph{Monte Carlo}.

\subsubsection{Algoritmi Monte Carlo}

Un algoritm Monte Carlo se caracterizează prin următoarele proprietăți:
\begin{itemize}
  \item timpul de execuție este determinist, fiind fixat în prealabil (de exemplu, printr-un număr $N$ de simulări);
  \item rezultatul obținut este aproximativ, nu exact;
  \item există o probabilitate nenulă de eroare, însă aceasta scade pe măsură ce crește numărul de simulări;
  \item precizia rezultatului depinde de legea numerelor mari.
\end{itemize}

În simulările realizate:
\begin{itemize}
  \item se execută un număr fix de aruncări aleatoare ale acului;
  \item se estimează o probabilitate geometrică sau valoarea constantei $\pi$;
  \item rezultatul nu este garantat exact, dar converge către valoarea teoretică atunci când $N \to \infty$.
\end{itemize}

Aceste caracteristici corespund exact definiției unui algoritm Monte Carlo.

\subsubsection{De ce strategia NU este de tip Las Vegas}

Un algoritm de tip Las Vegas are proprietăți diferite:
\begin{itemize}
  \item rezultatul furnizat este întotdeauna corect;
  \item timpul de execuție este aleator, nefiind cunoscut în avans.
\end{itemize}

În cazul simulărilor analizate:
\begin{itemize}
  \item nu se garantează obținerea valorii exacte (de exemplu, $\pi$ nu este determinat exact);
  \item algoritmul se oprește după un număr fix de pași, deci timpul de execuție nu este aleator.
\end{itemize}

Prin urmare, strategia utilizată nu poate fi clasificată ca fiind de tip Las Vegas.

\subsubsection{Exemplu de algoritm Las Vegas}

Un exemplu clasic de algoritm Las Vegas este căutarea aleatoare a unui element într-un vector: se aleg poziții aleator până când elementul dorit este găsit. Algoritmul se oprește doar atunci când rezultatul corect este obținut, însă numărul de pași necesari este aleator.

\begin{verbatim}
las_vegas_search <- function(v, x) {
  n <- length(v)
  repeat {
    i <- sample(1:n, 1)
    if (v[i] == x) {
      return(i)
    }
  }
}
\end{verbatim}

În acest caz:
\begin{itemize}
  \item corectitudinea rezultatului este garantată;
  \item timpul de execuție nu este determinist.
\end{itemize}

\subsubsection{Concluzie}

Strategia aleatoare utilizată în simulările de tip Buffon este una Monte Carlo, deoarece algoritmul rulează un număr fix de pași și furnizează o estimare probabilistică a rezultatului, a cărei precizie depinde de numărul de simulări efectuate.

În contrast, algoritmii de tip Las Vegas garantează corectitudinea rezultatului, dar au un timp de execuție aleator.

\subsection{Observații}

Metoda Monte Carlo oferă o abordare intuitivă și flexibilă pentru estimarea unor mărimi teoretice dificil de calculat exact prin alte metode. Deși rezultatele nu sunt exacte, precizia acestora poate fi controlată prin creșterea numărului de simulări.

Un aspect important este compromisul dintre timpul de execuție și precizie: un număr mai mare de simulări conduce la rezultate mai bune, dar necesită resurse de calcul mai mari.

În concluzie, exercițiul demonstrează eficiența algoritmilor Monte Carlo în probleme de probabilitate geometrică și evidențiază diferența fundamentală față de algoritmii de tip Las Vegas, care garantează corectitudinea rezultatului, dar nu au timp de execuție determinist.

% ------------------------------------------------------------------------------
% PACHETE & SURSE
% ------------------------------------------------------------------------------

\sectiune{Pachete \& Surse Bibliografice}
\label{sec:pachete}

\subsection{Pachete R Utilizate}

\subsubsection{Persoana A -- Exercițiul 1}

\textit{Completat de Persoana A}

[Listă pachete pentru exercițiul 1...]

\subsubsection{Moga Eduard-Andrei -- Exercițiul 2}

\begin{itemize}
    \item \textbf{shiny} (versiunea $\geq$ 1.7.0) -- Framework pentru aplicații web interactive în R
    \begin{itemize}
        \item \textit{Utilizat pentru:} Crearea interfeței grafice, gestionarea reactivității, rendering grafice dinamice
        \item \textit{Funcții principale:} \texttt{fluidPage()}, \texttt{renderPlot()}, \texttt{reactive()}, \texttt{eventReactive()}
    \end{itemize}
    
    \item \textbf{stats} (pachet base R) -- Funcții statistice și de distribuții
    \begin{itemize}
        \item \textit{Utilizat pentru:} Generare eșantioane aleatorii, calcul statistici descriptive
        \item \textit{Funcții principale:} \texttt{rnorm()}, \texttt{rexp()}, \texttt{rpois()}, \texttt{rbinom()}, \texttt{ecdf()}, \texttt{quantile()}
    \end{itemize}
    
    \item \textbf{graphics} (pachet base R) -- Funcții grafice
    \begin{itemize}
        \item \textit{Utilizat pentru:} Crearea histogramelor, grafice ECDF, Q-Q plots
        \item \textit{Funcții principale:} \texttt{hist()}, \texttt{plot()}, \texttt{qqnorm()}, \texttt{qqline()}, \texttt{abline()}
    \end{itemize}
\end{itemize}


\subsection{Surse Bibliografice și Documentație}

\subsubsection{Persoana A}

\textit{Completat de Persoana A}

[Surse pentru exercițiul 1...]

\subsubsection{Moga Eduard-Andrei}

\begin{enumerate}
    \item \textbf{Documentație oficială Shiny}
    \begin{itemize}
        \item URL: \url{https://shiny.posit.co/}
        \item Tutorial complet: \url{https://shiny.rstudio.com/tutorial/}
        \item Utilizat pentru înțelegerea structurii aplicațiilor reactive și best practices
    \end{itemize}
    
    \item \textbf{Curs Probabilități și Statistică FMI}
    \begin{itemize}
        \item URL: \url{https://alexamarioarei.quarto.pub/curs-ps-fmi/}
        \item Secțiunile relevante: Introducere în R, Distribuții de probabilitate, Grafică
        \item Utilizat pentru: Concepte teoretice despre distribuții și transformări
    \end{itemize}
    
    \item \textbf{R Documentation}
    \begin{itemize}
        \item URL: \url{https://www.rdocumentation.org/}
        \item Utilizat pentru: Consultarea documentației funcțiilor statistice și grafice
    \end{itemize}
    
    \item \textbf{Shiny Gallery}
    \begin{itemize}
        \item URL: \url{https://shiny.rstudio.com/gallery/}
        \item Utilizat pentru: Inspirație în design UI și structură aplicație
    \end{itemize}
    
    \item \textbf{StackOverflow - R și Shiny}
    \begin{itemize}
        \item Utilizat pentru: Rezolvarea problemelor tehnice specifice (ECDF plotting, reactive UI)
    \end{itemize}
\end{enumerate}

\subsubsection{Tirdea Dominic Alexandru -  Exercițiul 3}
\begin{enumerate}
    \item \textbf{Grinstead, C. M., Snell, J. L. -- Introduction to Probability}
    \begin{itemize}
        \item URL: \url{https://math.dartmouth.edu/~prob/prob/prob.pdf}
        \item Instituție: Dartmouth College
        \item Utilizat pentru: capitole despre probabilitate geometrică și metode Monte Carlo
    \end{itemize}
    
    \item \textbf{Wikipedia -- Monte Carlo method}
    \begin{itemize}
        \item URL: \url{https://en.wikipedia.org/wiki/Monte_Carlo_method}
        \item Utilizat pentru: prezentare generală a metodelor Monte Carlo și aplicații
    \end{itemize}
    
    \item \textbf{Wikipedia -- Las Vegas algorithm}
    \begin{itemize}
        \item URL: \url{https://en.wikipedia.org/wiki/Las_Vegas_algorithm}
        \item Utilizat pentru: clasificarea algoritmilor aleatori și comparația Monte Carlo vs.\ Las Vegas
    \end{itemize}
    
    \item \textbf{R Documentation}
    \begin{itemize}
        \ietm URL: \url{https://www.rdocumentation.org/}
        \item Utilizat pentru dezvoltarea simularilor
    \end{itemize}
    
\end{enumerate}

% ------------------------------------------------------------------------------
% DIFICULTĂȚI
% ------------------------------------------------------------------------------

\sectiune{Dificultăți Întâmpinate}
\label{sec:dificultati}

\subsection{Persoana A}

\textit{Completat de Persoana A}

\begin{itemize}
    \item [Dificultate 1]
    \item [Dificultate 2]
    \item [Dificultate 3]
    \item [Dificultate 4]
\end{itemize}

\subsection{Moga Eduard-Andrei}

\begin{itemize}
    \item \textbf{Implementarea dinamică a transformărilor specifice fiecărei distribuții}
    
    Fiecare distribuție din enunț avea transformări specifice diferite (de exemplu, Normal avea $3+2X$, Exponențială avea $2-5X$, etc.). Am întâmpinat dificultăți în:
    \begin{itemize}
        \item Crearea unui sistem modular care să permită aplicarea automată a transformării corecte
        \item Parsarea coeficienților din stringurile de transformare
        \item Sincronizarea UI-ului pentru a afișa doar transformările valide pentru distribuția selectată
    \end{itemize}
    \textit{Soluție:} Am creat o funcție \texttt{apply\_transformation()} care primește ca parametru tipul de transformare ca string și aplică logica corespunzătoare folosind structuri conditionale.
    
    \item \textbf{Gestionarea corectă a dimensiunii eșantionului pentru sume de variabile}
    
    Pentru transformările de tipul $\sum_{i=1}^n X_i$, trebuia să generez $n \times m$ valori (unde $m$ este numărul de variabile de sumat), apoi să le grupez și să le sumez corect. Dificultățile au fost:
    \begin{itemize}
        \item Dimensionarea corectă a vectorului de date brute
        \item Împărțirea în matrice și sumarea pe linii/coloane
        \item Validarea că lungimea finală a vectorului transformat este corectă
    \end{itemize}
    \textit{Soluție:} Am folosit funcția \texttt{matrix()} pentru restructurare și \texttt{rowSums()} pentru agregare eficientă.
    
    \item \textbf{Sincronizarea UI-ului reactiv cu parametrii distribuțiilor}
    
    Fiecare distribuție are parametri diferiți (Normal are $\mu$ și $\sigma$, Poisson doar $\lambda$, etc.). Am întâmpinat probleme în:
    \begin{itemize}
        \item Afișarea dinamică a controalelor UI doar pentru parametrii relevanți
        \item Prevenirea erorilor când utilizatorul schimbă rapid distribuția
        \item Păstrarea valorilor default rezonabile pentru fiecare parametru
    \end{itemize}
    \textit{Soluție:} Am folosit \texttt{renderUI()} pentru a crea controale dinamice și \texttt{req()} pentru a preveni evaluarea prematură a expresiilor reactive.
    
    \item \textbf{Optimizarea performanței pentru eșantioane mari (> 1000 observații)}
    
    Pentru $n = 5000$, aplicația devenea lentă la regenerarea graficelor. Problemele identificate:
    \begin{itemize}
        \item Recalcularea inutilă a statisticilor la fiecare render
        \item Desenarea histogramei cu prea multe bin-uri
        \item ECDF computațional costisitoare pentru multe puncte
    \end{itemize}
    \textit{Soluție:} Am folosit \texttt{eventReactive()} în loc de \texttt{reactive()} pentru a controla exact când se regenerează datele și am optimizat numărul de breaks în histogramă.
\end{itemize}

\subsection{Tirdea Dominic-Alexandru}

\begin{itemize}
  \item \textbf{Înțelegerea modelului geometric și a variabilelor aleatoare}

    O primă dificultate a constat în înțelegerea corectă a modelului geometric și a variabilelor aleatoare implicate. 
    \begin{itemize}
      \item Interpretarea geometrică a poziției acului față de drepte
      \item Corelarea unghiului de orientare cu proiecțiile trigonometrice
      \item Legătura dintre configurația geometrică și condițiile matematice de intersecție
    \end{itemize}
    \textit{Soluție:} Am analizat fiecare configurație geometrică separat și am formulat explicit condițiile de intersecție folosind proiecții trigonometrice și argumente de simetrie.

  \item \textbf{Formularea corectă a condițiilor de intersecție}

    O altă dificultate importantă a fost formularea corectă a condițiilor de intersecție pentru fiecare configurație analizată (acul clasic, crucea, gridul bidimensional).
    \begin{itemize}
      \item Identificarea proiecțiilor relevante ale acului pe direcțiile normale
      \item Tratarea corectă a cazurilor cu mai multe direcții de intersecție
      \item Asigurarea coerenței între modele diferite
    \end{itemize}
    \textit{Soluție:} Am derivat analitic condițiile de intersecție pentru fiecare caz și le-am verificat prin comparație cu formulele teoretice cunoscute.

  \item \textbf{Generarea corectă a variabilelor aleatoare}

    Din punct de vedere al implementării, o provocare majoră a fost alegerea unei simulări aleatoare corecte, care să respecte ipotezele de uniformitate și independență ale variabilelor.
    \begin{itemize}
      \item Generarea uniformă a poziției centrului acului
      \item Generarea uniformă a unghiului de orientare
      \item Evitarea erorilor de interpretare a intervalelor unghiulare
    \end{itemize}
    \textit{Soluție:} Am utilizat funcții standard pentru generarea variabilelor uniforme și am verificat distribuțiile obținute prin teste și reprezentări grafice.

\end{itemize}

% ------------------------------------------------------------------------------
% PROBLEME DESCHISE
% ------------------------------------------------------------------------------

\sectiune{Probleme Deschise}
\label{sec:probleme}

\subsection{Persoana A}

\textit{Completat de Persoana A}

\begin{itemize}
    \item [Problemă deschisă 1]
    \item [Problemă deschisă 2]
    \item [Problemă deschisă 3]
\end{itemize}

\subsection{Moga Eduard-Andrei}

\begin{itemize}
    \item \textbf{Extinderea aplicației cu distribuții suplimentare}
    
    În versiunea actuală sunt implementate 5 distribuții conform cerințelor. O îmbunătățire viitoare ar fi adăugarea de noi distribuții precum:
    \begin{itemize}
        \item Distribuția Gamma și Beta (pentru modelare mai flexibilă)
        \item Distribuția Uniformă (pentru comparație cu celelalte)
        \item Distribuția $t$-Student și Chi-pătrat (pentru statistică inferențială)
        \item Distribuții discrete: Geometrică, Binomială Negativă
    \end{itemize}
    Implementarea ar necesita extinderea funcției \texttt{gen\_sample()} și adăugarea transformărilor specifice pentru fiecare distribuție nouă.
    
    \item \textbf{Comparație vizuală între distribuția teoretică și ECDF}
    
    Deși aplicația calculează și afișează ECDF, nu există o suprapunere directă cu funcția de repartiție teoretică. Ar fi util să se implementeze:
    \begin{itemize}
        \item Calculul și afișarea funcției $F(x)$ teoretice pe același grafic cu ECDF
        \item Calcul distanță Kolmogorov-Smirnov: $D_n = \sup_x |F_n(x) - F(x)|$
        \item Test formal pentru verificarea adecvării la distribuție
        \item Măsurarea convergenței ECDF către $F$ în funcție de $n$
    \end{itemize}
    
    \item \textbf{Funcționalitate de export și salvare a rezultatelor}
    
    Momentan, utilizatorul poate vizualiza rezultatele doar în browser. Îmbunătățiri viitoare:
    \begin{itemize}
        \item Export grafice în format PNG/PDF de înaltă rezoluție
        \item Export tabel cu statistici în format CSV/Excel
        \item Salvarea configurației curente (distribuție, parametri, transformare)
        \item Generarea automată de rapoarte HTML cu toate rezultatele
        \item Compararea mai multor configurații side-by-side
    \end{itemize}
    Implementarea ar putea folosi pachete precum \texttt{downloadHandler()} din Shiny și \texttt{rmarkdown} pentru rapoarte.
\end{itemize}

\subsection{Tirdea Dominic-Alexandru }


\begin{itemize}
    \item \textbf{Analiza erorilor}
    \begin{itemize}
      O primă problemă deschisă este legată de analiza erorii estimărilor obținute prin metoda Monte Carlo. În cadrul proiectului, s-a observat empiric convergența rezultatelor către valorile teoretice, însă nu a fost realizată o evaluare riguroasă a vitezei de convergență sau a intervalelor de încredere asociate estimărilor. O extindere naturală ar fi studiul variației erorii în funcție de numărul de simulări și compararea acesteia cu estimările teoretice oferite de legea numerelor mari sau de teorema limitei centrale.
    \end{itemize}

    \item \textbf{Generalizarea configurațiilor}
    \begin{itemize}
      O altă direcție de interes o reprezintă generalizarea configurațiilor geometrice. În acest proiect au fost analizate doar cazurile clasice (linii paralele, cruce, grid dreptunghiular). Ar putea fi studiate configurații mai complexe, cum ar fi grile neuniforme, linii dispuse sub unghiuri arbitrare sau obiecte cu forme diferite de ac (segmente curbe, poligoane).
    \end{itemize}

    \item \textbf{Optimizare algoritmilor}
    \begin{itemize}
      De asemenea, o problemă deschisă este optimizarea algoritmică a simulărilor. Metoda Monte Carlo utilizată este una directă (brute-force), fără tehnici de reducere a varianței. Introducerea unor metode precum importance sampling sau stratificarea domeniului ar putea conduce la estimări mai precise pentru același timp de execuție.1
    \end{itemize}

    
\end{itemize}

% ------------------------------------------------------------------------------
% CONCLUZII
% ------------------------------------------------------------------------------

\sectiune{Concluzii}
\label{sec:concluzii}

\textit{Această secțiune este completată de lider (Persoana A).}

\vspace{1cm}

[Aici liderul scrie concluziile generale ale proiectului, sintetizând rezultatele obținute de toate cele trei exerciții și reflectând asupra procesului de colaborare și învățare.]

Principalele contribuții ale proiectului:
\begin{itemize}
    \item Implementarea practică a conceptelor teoretice studiate la curs
    \item Dezvoltarea abilităților de programare în R
    \item Înțelegerea profundă a comportamentului distribuțiilor de probabilitate
    \item Colaborare eficientă în echipă și organizare structurată a codului
\end{itemize}

Proiectul demonstrează aplicabilitatea metodelor de simulare Monte Carlo în verificarea rezultatelor teoretice și oferă instrumente interactive pentru explorarea conceptelor de probabilitate și statistică.

% ------------------------------------------------------------------------------
% BIBLIOGRAFIE (opțional)
% ------------------------------------------------------------------------------

\newpage
\begin{thebibliography}{99}

\bibitem{shiny}
Chang, W., Cheng, J., Allaire, J., et al. (2023). 
\textit{shiny: Web Application Framework for R}. 
R package version 1.7.5. 
\url{https://CRAN.R-project.org/package=shiny}

\bibitem{rcoreteam}
R Core Team (2023). 
\textit{R: A Language and Environment for Statistical Computing}. 
R Foundation for Statistical Computing, Vienna, Austria. 
\url{https://www.R-project.org/}

\bibitem{cursps}
Amărioarei, A. (2025). 
\textit{Curs Probabilități și Statistică 2025-2026}. 
Facultatea de Matematică și Informatică, Universitatea din București.
\url{https://alexamarioarei.quarto.pub/curs-ps-fmi/}

\end{thebibliography}

% ------------------------------------------------------------------------------
% ANEXE (opțional)
% ------------------------------------------------------------------------------

\newpage
\appendix

\section{Cod Sursă Exercițiul 2}
\label{app:cod}

\textit{Codul complet al aplicației Shiny se găsește în fișierul \texttt{ex2\_shiny/app.R}.}

Pentru consultare detaliată, vezi:
\begin{itemize}
    \item Repository: \texttt{/ex2\_shiny/}
    \item Documentație: \texttt{/ex2\_shiny/README.md}
    \item Cod principal: \texttt{/ex2\_shiny/app.R}
\end{itemize}

\end{document}
